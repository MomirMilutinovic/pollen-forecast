% Options for packages loaded elsewhere
\PassOptionsToPackage{unicode}{hyperref}
\PassOptionsToPackage{hyphens}{url}
%
\documentclass[
]{article}
\usepackage{lmodern}
\usepackage{amssymb,amsmath}
\usepackage{ifxetex,ifluatex}
\ifnum 0\ifxetex 1\fi\ifluatex 1\fi=0 % if pdftex
  \usepackage[T1]{fontenc}
  \usepackage[utf8]{inputenc}
  \usepackage{textcomp} % provide euro and other symbols
\else % if luatex or xetex
  \usepackage{unicode-math}
  \defaultfontfeatures{Scale=MatchLowercase}
  \defaultfontfeatures[\rmfamily]{Ligatures=TeX,Scale=1}
\fi
% Use upquote if available, for straight quotes in verbatim environments
\IfFileExists{upquote.sty}{\usepackage{upquote}}{}
\IfFileExists{microtype.sty}{% use microtype if available
  \usepackage[]{microtype}
  \UseMicrotypeSet[protrusion]{basicmath} % disable protrusion for tt fonts
}{}
\makeatletter
\@ifundefined{KOMAClassName}{% if non-KOMA class
  \IfFileExists{parskip.sty}{%
    \usepackage{parskip}
  }{% else
    \setlength{\parindent}{0pt}
    \setlength{\parskip}{6pt plus 2pt minus 1pt}}
}{% if KOMA class
  \KOMAoptions{parskip=half}}
\makeatother
\usepackage{xcolor}
\IfFileExists{xurl.sty}{\usepackage{xurl}}{} % add URL line breaks if available
\IfFileExists{bookmark.sty}{\usepackage{bookmark}}{\usepackage{hyperref}}
\hypersetup{
  pdftitle={Polen Open Data},
  pdfauthor={Momir Milutinovic},
  hidelinks,
  pdfcreator={LaTeX via pandoc}}
\urlstyle{same} % disable monospaced font for URLs
\usepackage[margin=1in]{geometry}
\usepackage{color}
\usepackage{fancyvrb}
\newcommand{\VerbBar}{|}
\newcommand{\VERB}{\Verb[commandchars=\\\{\}]}
\DefineVerbatimEnvironment{Highlighting}{Verbatim}{commandchars=\\\{\}}
% Add ',fontsize=\small' for more characters per line
\usepackage{framed}
\definecolor{shadecolor}{RGB}{248,248,248}
\newenvironment{Shaded}{\begin{snugshade}}{\end{snugshade}}
\newcommand{\AlertTok}[1]{\textcolor[rgb]{0.94,0.16,0.16}{#1}}
\newcommand{\AnnotationTok}[1]{\textcolor[rgb]{0.56,0.35,0.01}{\textbf{\textit{#1}}}}
\newcommand{\AttributeTok}[1]{\textcolor[rgb]{0.77,0.63,0.00}{#1}}
\newcommand{\BaseNTok}[1]{\textcolor[rgb]{0.00,0.00,0.81}{#1}}
\newcommand{\BuiltInTok}[1]{#1}
\newcommand{\CharTok}[1]{\textcolor[rgb]{0.31,0.60,0.02}{#1}}
\newcommand{\CommentTok}[1]{\textcolor[rgb]{0.56,0.35,0.01}{\textit{#1}}}
\newcommand{\CommentVarTok}[1]{\textcolor[rgb]{0.56,0.35,0.01}{\textbf{\textit{#1}}}}
\newcommand{\ConstantTok}[1]{\textcolor[rgb]{0.00,0.00,0.00}{#1}}
\newcommand{\ControlFlowTok}[1]{\textcolor[rgb]{0.13,0.29,0.53}{\textbf{#1}}}
\newcommand{\DataTypeTok}[1]{\textcolor[rgb]{0.13,0.29,0.53}{#1}}
\newcommand{\DecValTok}[1]{\textcolor[rgb]{0.00,0.00,0.81}{#1}}
\newcommand{\DocumentationTok}[1]{\textcolor[rgb]{0.56,0.35,0.01}{\textbf{\textit{#1}}}}
\newcommand{\ErrorTok}[1]{\textcolor[rgb]{0.64,0.00,0.00}{\textbf{#1}}}
\newcommand{\ExtensionTok}[1]{#1}
\newcommand{\FloatTok}[1]{\textcolor[rgb]{0.00,0.00,0.81}{#1}}
\newcommand{\FunctionTok}[1]{\textcolor[rgb]{0.00,0.00,0.00}{#1}}
\newcommand{\ImportTok}[1]{#1}
\newcommand{\InformationTok}[1]{\textcolor[rgb]{0.56,0.35,0.01}{\textbf{\textit{#1}}}}
\newcommand{\KeywordTok}[1]{\textcolor[rgb]{0.13,0.29,0.53}{\textbf{#1}}}
\newcommand{\NormalTok}[1]{#1}
\newcommand{\OperatorTok}[1]{\textcolor[rgb]{0.81,0.36,0.00}{\textbf{#1}}}
\newcommand{\OtherTok}[1]{\textcolor[rgb]{0.56,0.35,0.01}{#1}}
\newcommand{\PreprocessorTok}[1]{\textcolor[rgb]{0.56,0.35,0.01}{\textit{#1}}}
\newcommand{\RegionMarkerTok}[1]{#1}
\newcommand{\SpecialCharTok}[1]{\textcolor[rgb]{0.00,0.00,0.00}{#1}}
\newcommand{\SpecialStringTok}[1]{\textcolor[rgb]{0.31,0.60,0.02}{#1}}
\newcommand{\StringTok}[1]{\textcolor[rgb]{0.31,0.60,0.02}{#1}}
\newcommand{\VariableTok}[1]{\textcolor[rgb]{0.00,0.00,0.00}{#1}}
\newcommand{\VerbatimStringTok}[1]{\textcolor[rgb]{0.31,0.60,0.02}{#1}}
\newcommand{\WarningTok}[1]{\textcolor[rgb]{0.56,0.35,0.01}{\textbf{\textit{#1}}}}
\usepackage{graphicx,grffile}
\makeatletter
\def\maxwidth{\ifdim\Gin@nat@width>\linewidth\linewidth\else\Gin@nat@width\fi}
\def\maxheight{\ifdim\Gin@nat@height>\textheight\textheight\else\Gin@nat@height\fi}
\makeatother
% Scale images if necessary, so that they will not overflow the page
% margins by default, and it is still possible to overwrite the defaults
% using explicit options in \includegraphics[width, height, ...]{}
\setkeys{Gin}{width=\maxwidth,height=\maxheight,keepaspectratio}
% Set default figure placement to htbp
\makeatletter
\def\fps@figure{htbp}
\makeatother
\setlength{\emergencystretch}{3em} % prevent overfull lines
\providecommand{\tightlist}{%
  \setlength{\itemsep}{0pt}\setlength{\parskip}{0pt}}
\setcounter{secnumdepth}{-\maxdimen} % remove section numbering

\title{Polen Open Data}
\author{Momir Milutinovic}
\date{2020-03-08}

\begin{document}
\maketitle

This is a data analysis report about the pollen levels in Belgrade with
the explanation of the R code used. Data that will be used for the
analysis comes from the
\href{http://polen.sepa.gov.rs/api/opendata/schema}{pollen API}, which
is maintained by the \href{http://sepa.gov.rs}{Environmental Protection
Agency}.

\hypertarget{reading-and-organising-data}{%
\section{Reading and organising
data}\label{reading-and-organising-data}}

We will start by reading in the data.

\begin{Shaded}
\begin{Highlighting}[]
\OperatorTok{>}\StringTok{ }\KeywordTok{suppressPackageStartupMessages}\NormalTok{(}\KeywordTok{library}\NormalTok{(dplyr))}
\OperatorTok{>}\StringTok{ }\KeywordTok{suppressPackageStartupMessages}\NormalTok{(}\KeywordTok{library}\NormalTok{(ggplot2))}
\OperatorTok{>}\StringTok{ }\KeywordTok{suppressPackageStartupMessages}\NormalTok{(}\KeywordTok{library}\NormalTok{(leaflet))}
\OperatorTok{>}\StringTok{ }
\ErrorTok{>}\StringTok{ }\NormalTok{mydata <{-}}\StringTok{ }\KeywordTok{read.csv}\NormalTok{(}\StringTok{"pollenData2019.csv"}\NormalTok{,}
\OperatorTok{+}\StringTok{                    }\DataTypeTok{header =}\NormalTok{ T,}
\OperatorTok{+}\StringTok{                    }\DataTypeTok{stringsAsFactors =} \OtherTok{FALSE}
\OperatorTok{+}\StringTok{                    }\NormalTok{)}
\end{Highlighting}
\end{Shaded}

Let's what we have here.

\begin{Shaded}
\begin{Highlighting}[]
\OperatorTok{>}\StringTok{ }\KeywordTok{glimpse}\NormalTok{(mydata)}
\end{Highlighting}
\end{Shaded}

\begin{verbatim}
## Observations: 37,809
## Variables: 9
## $ X              <int> 1, 2, 3, 4, 5, 6, 7, 8, 9, 10, 11, 12, 13, 14, 15, 16,…
## $ id             <int> 13167, 13174, 13175, 13176, 13176, 13177, 13177, 13178…
## $ date           <chr> "2019-03-11", "2019-02-11", "2019-02-12", "2019-02-14"…
## $ location_name  <chr> "БЕОГРАД - ЗЕЛЕНО БРДО", "ЧАЧАК", "ЧАЧАК", "ЧАЧАК", "Ч…
## $ lat            <dbl> 44.8000, 43.8927, 43.8927, 43.8927, 43.8927, 43.8927, …
## $ long           <dbl> 20.46667, 20.34436, 20.34436, 20.34436, 20.34436, 20.3…
## $ value          <int> 2655, 1, 1, 3, 1, 4, 1, 1, 1, 1, 10, 1, 18, 20, 88, 2,…
## $ allergen_name  <chr> "MORACEAE", "CORYLUS", "CORYLUS", "CORYLUS", "PINACEAE…
## $ localized_name <chr> "ДУД", "ЛЕСКА", "ЛЕСКА", "ЛЕСКА", "ЧЕТИНАРИ", "ЛЕСКА",…
\end{verbatim}

We don't really need the X and id columns at all.

\begin{Shaded}
\begin{Highlighting}[]
\OperatorTok{>}\StringTok{ }\NormalTok{mydata <{-}}\StringTok{ }\NormalTok{mydata[,}\OperatorTok{{-}}\KeywordTok{c}\NormalTok{(}\DecValTok{1}\NormalTok{, }\DecValTok{2}\NormalTok{)]}
\OperatorTok{>}\StringTok{ }\KeywordTok{glimpse}\NormalTok{(mydata)}
\end{Highlighting}
\end{Shaded}

\begin{verbatim}
## Observations: 37,809
## Variables: 7
## $ date           <chr> "2019-03-11", "2019-02-11", "2019-02-12", "2019-02-14"…
## $ location_name  <chr> "БЕОГРАД - ЗЕЛЕНО БРДО", "ЧАЧАК", "ЧАЧАК", "ЧАЧАК", "Ч…
## $ lat            <dbl> 44.8000, 43.8927, 43.8927, 43.8927, 43.8927, 43.8927, …
## $ long           <dbl> 20.46667, 20.34436, 20.34436, 20.34436, 20.34436, 20.3…
## $ value          <int> 2655, 1, 1, 3, 1, 4, 1, 1, 1, 1, 10, 1, 18, 20, 88, 2,…
## $ allergen_name  <chr> "MORACEAE", "CORYLUS", "CORYLUS", "CORYLUS", "PINACEAE…
## $ localized_name <chr> "ДУД", "ЛЕСКА", "ЛЕСКА", "ЛЕСКА", "ЧЕТИНАРИ", "ЛЕСКА",…
\end{verbatim}

That's better! We will check how many unique records each variable in
our data has.

\begin{Shaded}
\begin{Highlighting}[]
\OperatorTok{>}\StringTok{ }\NormalTok{(uniq <{-}}\StringTok{ }\KeywordTok{unlist}\NormalTok{(}\KeywordTok{lapply}\NormalTok{(mydata, }\ControlFlowTok{function}\NormalTok{(x) }\KeywordTok{length}\NormalTok{(}\KeywordTok{unique}\NormalTok{(x)))))}
\end{Highlighting}
\end{Shaded}

\begin{verbatim}
##           date  location_name            lat           long          value 
##            281             26             22             22            564 
##  allergen_name localized_name 
##             26             26
\end{verbatim}

There are 26 stations, of which only two are in Belgrade. We can see
that there are more stations than unique geographic coordinate values.
This is due to the location of some stations being 0,0 for reasons
beyond our control. They are on the second page.

\begin{Shaded}
\begin{Highlighting}[]
\OperatorTok{>}\StringTok{ }\KeywordTok{print}\NormalTok{(}\KeywordTok{unique}\NormalTok{(mydata[, }\KeywordTok{c}\NormalTok{(}\StringTok{"location\_name"}\NormalTok{, }\StringTok{"lat"}\NormalTok{, }\StringTok{"long"}\NormalTok{)]))}
\end{Highlighting}
\end{Shaded}

\begin{verbatim}
##                location_name      lat     long
## 1      БЕОГРАД - ЗЕЛЕНО БРДО 44.80000 20.46667
## 2                      ЧАЧАК 43.89270 20.34436
## 109                 КРУШЕВАЦ 43.58157 21.32034
## 209                  ЛОЗНИЦА 44.53713 19.22986
## 586   БЕОГРАД - НОВИ БЕОГРАД 44.82317 20.41291
## 702                   ВАЉЕВО 44.27088 19.88723
## 799                  ПАНЧЕВО 44.86823 20.65233
## 1052               ОБРЕНОВАЦ 44.65818 20.20004
## 1284                   БЕЧЕЈ 45.61306 20.05277
## 1475                СУБОТИЦА 46.10497 19.66857
## 1552                     НИШ 43.31629 21.91368
## 1834       СРЕМСКА МИТРОВИЦА  0.00000  0.00000
## 2619                 КИКИНДА  0.00000  0.00000
## 2818                   ВРБАС  0.00000  0.00000
## 2897                ЗРЕЊАНИН  0.00000  0.00000
## 3414                  СОМБОР  0.00000  0.00000
## 3666                СОКОБАЊА 43.63953 21.86840
## 4055               ПОЖАРЕВАЦ 44.62116 21.18908
## 5803                   ВРШАЦ 45.11667 21.30361
## 8659                ЗЛАТИБОР 43.72186 19.70833
## 15258                ЗАЈЕЧАР 43.90631 22.27835
## 16357             НОВИ ПАЗАР 43.14081 20.51837
## 16547                  ВРАЊЕ 42.55146 21.90231
## 17720                   КУЛА 45.70159 19.38150
## 17823             КРАГУЈЕВАЦ 44.01181 20.91589
## 17868                КРАЉЕВО 43.72399 20.68891
\end{verbatim}

However, none of them are in Belgrade. The two stations in Belgrade are
\emph{БЕОГРАД - ЗЕЛЕНО БРДО} and \emph{БЕОГРАД - НОВИ БЕОГРАД}.

\end{document}
